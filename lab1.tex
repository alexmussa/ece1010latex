\documentclass[12pt]{article}
\usepackage[english]{babel}
\usepackage[utf8x]{inputenc}
\usepackage[T1]{fontenc}
\usepackage{lab}
\usepackage{listings}
\usepackage{url}


\Instructors{Alex Mussa, Kevin Johnson}
\LabNumber{1}
\LabTitle{ECE Subdiscipline Research}
\LabDate{June 24th, 2019}

\lstset{style=mystyle}

\begin{document}
\MakeLabTop

\section{The Sub-disciplines of Electrical and Computer Engineering}

Electrical and Computer Engineering (ECE) covers an enormous breadth of topics, which necessitates specialization into smaller fields that focus on a narrower field. Here at the Georgia Institute of Technology, ECE is divided into 11 different categories, referred to as Technical Interest Groups (TIGs) that collectively represent the major research areas of the School and the topics of courses. These areas are highly interrelated, and many faculty members engage in research in multiple TIGs and many courses also pull from topics spanning two or more TIGs. The following subsections provide a brief statement about each TIG, as well as some major topics of interest for each.

\newpage

\subsection{Bioengineering}

Bioengineering is concerned with the application of engineering principles to the study and control of biological processes. In this area, mathematical and physical concepts are developed which are applied to medicine and biology. Specific applications include feature extraction in cardiac imagery, Micro-electro-mechanical Systems (MEMS) devices for direct interfacing with biological systems, modeling of biological sensory and motor systems, and the development of sensors for the detection of cancer cells. Because of the continued advancement of medical technology and the many unsolved problems in the understanding and treatment of disease, bioengineers play an essential role in the improvement of the understanding of biological systems, and in the development and evaluation of healthcare technologies.

Some topics of interest for Bioengineering include:

\begin{itemize}
    \item Biosensors/BioMEMS
    \item Neuroengineering
    \item Medical Imaging and Signal Processing
\end{itemize}

Academically, some of the courses that focus on Bioengineering principles include those listed at \url{https://www.ece.gatech.edu/courses/course_list_tig/5}.

\subsection{Computer Systems and Software}

Computer Systems and Software (CSS) carries out activities that span from high performance microarchitecture to integrated hardware/software systems to large-scale distributed software and internetworks. Computer systems and software activities focus on the optimization of cross cutting characteristics, such as power/energy consumption and security, within application domains such as health, high performance computing, and embedded real-time systems. Modern systems of all types rely heavily on software for their operation. Thus, it is becoming essential for all engineers to be familiar with modern software design and development techniques. In many application areas, hardware is now a commodity and software provides the "value-added" proposition.

Some topics of interest for Computer Systems and Software include:

\begin{itemize}
    \item Computer Architecture
    \item Embedded Systems
    \item Computer Networks and Internetworking
    \item System Software
    \item Computer System and Network Security
\end{itemize}

Academically, some of the courses that focus on CSS principles include those listed at \url{https://www.ece.gatech.edu/courses/course_list_tig/10}.

\subsection{Digital Signal Processing}

Digital Signal Processing (DSP) is concerned with the representation of signals in digital form, and with the transformation of such signal representations using digital computation.

Digital Signal Processing is at the core of virtually all of today's information technology, and its impact is felt everywhere -- in telecommunications, medical technology, radar and sonar, and in seismic data analysis. The Digital Signal Processing Group at ECE operates the largest educational and research programs in the world in both size and impact.

Some topics of interest for Digital Signal Processing include:

\begin{itemize}
    \item Image and Video Signal Processing
    \item Signal Processing for Communications and Security
    \item Radar and Array Processing
    \item Speech and Audio Processing
    \item Machine Learning
    \item Various Applications Related to Autonomy and Automation
\end{itemize}

Academically, some of the courses that focus on DSP principles include those listed at \url{https://www.ece.gatech.edu/courses/course_list_tig/15}.

\subsection{Electrical Energy}

Electrical Energy is primarily concerned with meeting the future demand for the generation and distribution of electric energy while satisfying environmental constraints. Electric Energy Systems are primarily concerned with meeting the demand for electric energy in a safe, reliable, secure, cost effective, and environmentally friendly manner.

Some topics of interest for Electric Energy include:

\begin{itemize}
    \item Power System Protection, Control, Optimization, and Automation
    \item Distributed and Customer Generation
    \item Power Electronics
    \item High Voltage Engineering
    \item Dielectric Materials and Silicon-based Microchips and Microsystems
\end{itemize}

Academically, some of the courses that focus on Electric Energy principles include those listed at \url{https://www.ece.gatech.edu/courses/course_list_tig/20}.

\subsection{Electromagnetics}

Electromagnetics involves the study of Maxwell's equations and their application to the analysis and design of devices and systems. Maxwell's equations represent one of the most concise statements of the fundamentals of electricity and magnetism. It is from here that most of the working relationships in the field of electromagnetics are developed. Electromagnetics includes everything about radio waves: their generation, propagation, and interactions with materials, devices, and systems. The dynamic field of electromagnetics encompasses such far-reaching areas as microwave communications, antenna design, microwave millimeter engineering, and remote sensing.

Some topics of interest for Electromagnetics include:

\begin{itemize}
    \item Microwave Circuits
    \item Radar, Communication, and Navigation
    \item EM Wave Propagation Modeling
    \item Wireless Networks and Security
\end{itemize}

Academically, some of the courses that focus on Electromagnetics principles include those listed at \url{https://www.ece.gatech.edu/courses/course_list_tig/25}.

\subsection{Electronic Design and Applications}

Electronic Design and Applications (EDA) involves device and integrated circuit fabrication, circuit and system design and simulation, and instrumentation and testing techniques. EDA is a vital area of electrical engineering, encompassing the experimentation, design, modeling, simulation and analysis of single devices or circuits as well as complete signal processing systems.

Some topics of interest for EDA include:

\begin{itemize}
    \item MEMS Devices and Circuits
    \item Analog VLSI
    \item Radio-Frequency (Wireless) Integrated Circuits
    \item High-Speed Mixed-Signal Systems
    \item Power-Management Integrated Circuits
\end{itemize}

Academically, some of the courses that focus on EDA principles include those listed at \url{https://www.ece.gatech.edu/courses/course_list_tig/30}.

\subsection{Nanotechnology}

Nanotechnology is concerned with the design, analysis, growth, and fabrication of micron/sub-micron feature length devices. The invention of the transistor and the integrated circuit marked the genesis of microelectronics and set the stage for the unprecedented technological advances of the 20th century, which impacted virtually every aspect of modern life. Indeed, it is said that no invention in the history of humanity has so quickly spread throughout the world, or so profoundly pervaded so many aspects of human existence as the microprocessor. The field of nanotechnology has enabled such life-altering breakthroughs as implantable cardiac pacemakers, personal computers, wireless cellular telephones, optoelectronic-fiber networks, communication satellites and the Internet.

Some topics of interest for Nanotechnology include:

\begin{itemize}
    \item Microsystems and Nanosystems
    \item Photovoltaics
    \item Microelectronics Systems Packaging
    \item Manufacturing and Gigascale Integration
    \item Compound Semiconductors
\end{itemize}

Academically, some of the courses that focus on nanotechnology principles include those listed at \url{https://www.ece.gatech.edu/courses/course_list_tig/35}.

\subsection{Optics and Photonics}

Optics and Photonics involves the study of lasers, optics, and holography.
The field of optics and photonics involves the study of lasers, optical data processing, nonlinear optics, optical communications, optical computing, optical data storage, optical system design and holography. These are the fundamental enabling technologies for future high-speed communication systems including high performance telecommunication and data communication networks.

Some topics of interest for Optics and Photonics include:

\begin{itemize}
    \item Optical Communication Networks
    \item Imaging and Display
    \item Integrated Photonics and Optoelectronics
    \item Diffractive and Holographic Optics
    \item Metamaterials and Metasurfaces
\end{itemize}

Academically, some of the courses that focus on nanotechnology principles include those listed at \url{https://www.ece.gatech.edu/courses/course_list_tig/40}.

\subsection{Systems and Controls}

Systems and controls is concerned with mathematical and computational techniques for modeling, estimation, and control of systems and processes. The principal mission of control engineers is to design controllers for systems. Systems and Controls provides the field of electrical and computer engineering a unified paradigm for designing controllers in a variety of application domains. Initially developed in the context of circuits, subsequent applications were primarily in weapons and aerospace industries. This dynamic field has come to represent an essential enabling and supporting technology for the field of electrical engineering with current applications ranging from defense and manufacturing to telecommunications and bioengineering.

Some topics of interest for Systems and Controls include:

\begin{itemize}
    \item Mathematical Systems Theory
    \item Discrete Event Systems and Hybrid Systems
    \item Nonlinear Control
    \item Sensor Technologies
    \item Robotics
\end{itemize}

Academically, some of the courses that focus on Systems and Controls principles include those listed at \url{https://www.ece.gatech.edu/courses/course_list_tig/45}.

\subsection{Telecommunications}

Telecommunications is concerned with the characterization, representation, transmission, storage, and networking of information over various media including space, optical fiber, and cable. Telecommunications encompasses several key areas of electrical engineering: digital signal processing, computer engineering, controls, and optics. Harnessing the leading technologies in these areas, the dynamic field of telecommunications plays a defining role in the information technology revolution that we are experiencing today. Specific applications include mobile communications, wireless local area networks (WLANs), television, telephony, and radar.

Some topics of interest for Telecommunications include:

\begin{itemize}
    \item Wireless Communications and Networking
    \item Communication Theory
    \item Information Theory and Adaptive Systems
    \item Multimedia Networking
    \item Optical Networks
\end{itemize}

Academically, some of the courses that focus on Telecommunications principles include those listed at \url{https://www.ece.gatech.edu/courses/course_list_tig/50}.

\subsection{VLSI Systems and Digital Design}

The VLSI systems and digital design technical interest group carries out activities involved with designing and testing complex digital and mixed-signal electronic systems. These techniques optimize power, performance, and reliability metrics across a wide range of applications. The interests of faculty in this area span all levels of abstraction: embedded software and hardware/software co-design; design synthesis; physical design; algorithms for accurate electrical simulation of chips and packages; design of 3-D systems and design of reliable digital, mixed-signal, and RF electronics; and system/package co-design. Key applications include surveillance, robotics, multimedia, and cloud computing that are optimized for power and reliability across the algorithm-architecture-circuit levels.

Some topics of interest for VLSI Systems and Digital Design include:

\begin{itemize}
    \item Design of ultra low power circuits and systems
    \item Microarchitecture and memory design for performance, power, and reliability
    \item Design and test of Systems on Chips (SOCs) and Systems in Packages (SiPs)
    \item Low power/high-speed interconnect and design automation for physical design
    \item Data acquisition systems for vision/sensing applications
    \item Design of nanotechnology and other emerging computational fabrics
\end{itemize}

Academically, some of the courses that focus on VLSI Systems and Digital Design principles include those listed at \url{https://www.ece.gatech.edu/courses/course_list_tig/53}.

\section{Further Research}
After reading the above brief descriptions of the 11 subdisciplines in ECE at Georgia Tech, research further about ECE on the internet. Specifically focus on learning more about \textbf{2 subdisciplines of your choosing}. 

\subsection{Searching the Internet:}
Start by searching through the School of ECE's website at the above provided links to learn about the classes that are offered in those subdisciplines. Note down any keywords you notice in the course titles and catalog descriptions of the courses for each TIG. Then search those keywords as well as the name of the subdiscipline itself through a search engine such as google. Read things such as the definitions of the keywords, relevant Wikipedia articles, and other websites that explain the subdiscipline and keywords further.  

While searching through the internet, consider the following tasks for each of your two chosen subdisciplines.  These are not separate tasks; instead of trying to complete them sequentially, keep them all in mind and complete them as they come up in your research.
\begin{enumerate}
    \item Identify some topics, aspects, and/or applications of the subdisciplines that you find interesting.  Think about what makes those particularly appealing to you.  Perhaps it's cutting-edge technology, or it improves peoples lives, or it's lucrative, or it's something you yourself would enjoy using the result of.
    \item Identify some technologies or processes you have used or heard of that you think may have specifically utilized the subdisciplines, and consider how the subdisciplines could contribute to the operation of the technology. An example of this would be the technology of audio recording: the field of digital signal processing relates to how the analog signals are stored in a digital format, and how the digitized audio can be analyzed and manipulated. This example description was gathered from the information about \textit{Digital Signal Processing} on its corresponding Wikipedia article, as well as the Wikipedia article on \textit{Audio Signal Processing}
    \item Identify a major technological product such as automobiles, video game systems, personal computers, cell phones, assembly lines, robotics, etc., and elaborate on where and how your chosen subdisciplines are utilized in the technology.  Try to choose a product that utilizes both of your selected subdisciplines, but it is ok if only one applies.
\end{enumerate}

\subsection{Recording your Findings:}

With this information, write a short document describing your findings from the above tasks. Create an introduction, a body, and a conclusion as detailed below.

\begin{enumerate}
    \item \textit{Introduction:} The reader knows nothing about what this document is until you tell them, so start with a description of what is contained in the document.  It can literally start with "This document [...]." Then provide a small amount of background information to provide some context for the information in the body of your document.
    
    \item \textit{Body:} Include your findings from the tasks listed in section 2.1 above. Try to introduce each subdiscipline and results from the above tasks in such a way that the document is easy to read and flows smoothly from one idea to the next. Avoid simply listing off the findings of the tasks for each subdiscipline.
    
    Depending on the information you found, and on your personal preference, you could organize your document by discussing one of your chosen subdisciplines and then the other, or you could interleave the two as you sequentially discuss the topics and applications, or you could mix those overall organizations.  For example, you could organize your document by:
    \begin{enumerate}
        \item Comparing and contrasting the topics or aspects you found interesting about both subdisciplines.
        \item Comparing and contrasting the technologies you've heard of previously that utilized each subdiscipline.
        \item Discussing a major technological product that utilizes both subdisciplines.
    \end{enumerate}
    Or:
    \begin{enumerate}
        \item Discussing the topics or aspects you found interesting about the first subdiscipline along with some technology you've heard of that utilizes that subdiscipline.
        \item Discussing the topics or aspects you found interesting about the second subdiscipline along with some technology you've heard of that utilizes that subdiscipline.
        \item Discuss a major technological product that utilizes both subdisciplines.
    \end{enumerate}
    Note that both organizations include all of the required information, but present it in different forms.
    
    \item \textit{Conclusion:} Conclude the paper with a short summary of what was found and some key takeaways.  Relate your findings to the overall purpose of the document.
\end{enumerate}

\subsection{Paper Specifications}

\begin{itemize}
    \item The introduction, body, and conclusion should be no more than 2 pages. The cover sheet and references are not included in that.
    \item Use a 12pt. font size and 1.5 line spacing.
    \item Include a cover sheet with the assignment title, your name and GTID, the course number, and the submission date.
    \item Include the sources of the information as references at the end of the document.  You do not need to use a formal citation format.
    \item Use Microsoft Word, Google Docs, or another word processor.
\end{itemize}

\end{document}